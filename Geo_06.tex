\documentclass[11pt]{article}

\usepackage[utf8]{inputenc}
\usepackage[T1]{fontenc}
\usepackage[ngerman]{babel}
\usepackage{lmodern}
\usepackage[german=quotes]{csquotes}
\usepackage{amsmath}
\usepackage{amssymb}
\usepackage{amsthm}
\usepackage{graphicx}
\usepackage{scrpage2}
\usepackage{geometry}
\usepackage{tikz, tikz-3dplot}
\usepackage[bookmarks=true,
bookmarksopen=true,
bookmarksnumbered=false,
pdfstartpage=1,
baseurl=,
pdftitle={ },
pdfauthor={Pascal Gepperth},
pdfstartview={FitH},
pdfsubject={ },
pdfkeywords={ },
breaklinks=true,
colorlinks=true,
linkcolor=black,
anchorcolor=black,
citecolor=black,
filecolor=black,
menucolor=black,
pagecolor=black,
urlcolor=black ]{hyperref}
\newcommand{\de}{\mathrm{d}}
\newcommand{\dif}[1]{\frac{\mathrm{d}}{\mathrm{d}#1}}
\usepackage{enumerate}
\newcommand{\ray}[1]{\stackrel{\rightharpoonup}{#1}}
\usepackage{marginnote}
\let\marginpar\marginnote
\renewcommand*{\marginfont}{\footnotesize}
\renewcommand{\vec}{\textbf}

\geometry{left=3cm, right=3cm, top=3cm, bottom=3cm}

\pagestyle{scrheadings}
\ohead{Geometrie\\
	Blatt 6\\
	P.Gepperth, S.Jung\\
	Gruppe 4}

\begin{document}

\section*{Aufgabe 2}
Seien ein Punkt $ Z = (z_1,z_2,z_3) \in \mathbb{R}^3 $, eine Gerade $ L:t\mapsto p+tr $ und eine Ebene $ E: m_1x_1 + m_2x_x + m_3x_3 = b $ gegeben. Ferner gelte $ Z \notin E $ und $ L \cap E $ ein Punkt.
\begin{enumerate}
	\item Es sei $ X = (x_1,x_2,x_3) \in \mathbb{R}^3 $ beliebig. Dann ist
	\begin{equation*}
	\begin{aligned}
	L(X): t \mapsto \vec x + t \vec r
	\end{aligned}
	\end{equation*}
	die zu $ L $ parallel verschobene Gerade, sodass $ X \in L $. Da $ K \cap E $ ein Punkt ist, ist auch $ L(X) \cap E = \lbrace S_1 \rbrace $ genau ein Punkt. Der Schnittpunkt lässt sich wie folgt bestimmen:
	\begin{equation*}
	\begin{aligned}
	&&m_1(x_1+tr_1) +m_2(x_2+tr_2) + m_3(x_3+tr_3) &= b\\
	\Leftrightarrow && t (m_1r_1+ m_2r_2 + m_3r_3) &= b - m_1x_1 - m_2x_2 - m_3x_3\\
	\Leftrightarrow && t &= \frac{b - (m_1x_1 + m_2x_2 + m_3x_3)}{m_1r_1 + m_2r_2 + m_3r_3}\\
	&& &= \frac{b - \langle \vec m, \vec x \rangle}{\langle \vec m, \vec r \rangle}
	\end{aligned}
	\end{equation*}
	Dabei bezeichnet $ \langle \cdot,\cdot \rangle $ das Standardskalarprodukt und
	\begin{equation*}
	\begin{aligned}
	s_1 &= \vec x + \frac{b - \langle \vec m, \vec x \rangle}{\langle \vec m,\vec r \rangle} \cdot \vec r.
	\end{aligned}
	\end{equation*}
	ist die Projektion von $ X $ parallel zu $ L $ auf $ E $.
	\item Für eine Zentralprojektion auf $ E $ mit Zentrum $ Z \notin E $ wird zuerst eine Ausschlussebene $ H $ bestimmt. Für diese gilt
	\begin{equation*}
	\begin{aligned}
	E \parallel H
	\end{aligned}
	\end{equation*}
	und
	\begin{equation*}
	\begin{aligned}
	Z \in H.
	\end{aligned}
	\end{equation*}
	Damit folgt
	\begin{equation*}
	\begin{aligned}
	H = \lbrace \vec x \in \mathbb{R}^3 : m_1x_1 + m_2x_2 + m_3x_3 = m_1z_1 + m_2z_2 + m_3z_3 \rbrace
	\end{aligned}
	\end{equation*}
	als Ausschlussebene. Sei nun $ X = (x_1,x_2,x_3) \notin H $. Definiere
	\begin{equation*}
	\begin{aligned}
	Y : t \mapsto \begin{pmatrix}
	z_1 \\ z_2 \\ z_3
	\end{pmatrix} + t \cdot \underbrace{\begin{pmatrix}
	x_1 - z_1\\
	x_2 - z_2\\
	x_3 - z_3
	\end{pmatrix}}_{=:\vec v}.
	\end{aligned}
	\end{equation*}
	$ Y $ ist nicht parallel zu $ E $, folglich existiert ein eindeutiger Schnittpunkt $ Y \cap E = \lbrace S_2\rbrace $, der sich analog zum vorangegangenen berechnet. Es gilt
	\begin{equation*}
	\begin{aligned}
	s_2 &= \vec z + \frac{b - \langle \vec m,\vec x\rangle}{\langle \vec m, \vec v \rangle} \cdot \vec v\\
	&=\vec z + \frac{b - \langle \vec m,\vec x\rangle}{b - \langle \vec m, \vec z \rangle} \cdot \vec v
	\end{aligned}
	\end{equation*}
\end{enumerate}

\section*{MC}
\begin{enumerate}
	\item 
	\item richtig
	\item 
	\item 
	\item richtig
\end{enumerate}
\end{document}

