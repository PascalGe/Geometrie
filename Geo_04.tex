\documentclass[11pt]{article}

\usepackage[utf8]{inputenc}
\usepackage[T1]{fontenc}
\usepackage[ngerman]{babel}
\usepackage{lmodern}
\usepackage[german=quotes]{csquotes}
\usepackage{amsmath}
\usepackage{amssymb}
\usepackage{amsthm}
\usepackage{graphicx}
\usepackage{scrpage2}
\usepackage{geometry}
\usepackage{tikz, tikz-3dplot}
\usepackage[bookmarks=true,
bookmarksopen=true,
bookmarksnumbered=false,
pdfstartpage=1,
baseurl=,
pdftitle={ },
pdfauthor={Pascal Gepperth},
pdfstartview={FitH},
pdfsubject={ },
pdfkeywords={ },
breaklinks=true,
colorlinks=true,
linkcolor=black,
anchorcolor=black,
citecolor=black,
filecolor=black,
menucolor=black,
pagecolor=black,
urlcolor=black ]{hyperref}
\newcommand{\de}{\mathrm{d}}
\newcommand{\dif}[1]{\frac{\mathrm{d}}{\mathrm{d}#1}}
\usepackage{enumerate}
\newcommand{\ray}[1]{\stackrel{\rightharpoonup}{#1}}
\usepackage{marginnote}
\let\marginpar\marginnote
\renewcommand*{\marginfont}{\footnotesize}

\geometry{left=3cm, right=3cm, top=3cm, bottom=3cm}

\pagestyle{scrheadings}
\ohead{Geometrie\\
	Blatt 3\\
	P.Gepperth, S.Jung\\
	Gruppe 4}

\begin{document}

\section*{Aufgabe 2}
\subsection*{a)}
\subsection*{b)}
Seien
\begin{equation}
\begin{aligned}
AB \cong CD \label{eq:cong}
\end{aligned}
\end{equation}
und 
\begin{equation}
\begin{aligned}
AB \parallel CD. \label{eq:parallel} \marginpar{Sodass $ A,C $ \enquote{links}}
\end{aligned}
\end{equation}
\subsubsection*{Behauptung}
Es gilt $ AC \cong BD $ und $ AC \parallel BD $.
\subsubsection*{Beweis}
Mit (\ref{eq:parallel}) gilt nach (I.29)
\begin{equation*}
\begin{aligned}
\angle BAD \simeq \angle ADC.
\end{aligned}
\end{equation*}
Ferner gilt mit (\ref{eq:cong}) und der gemeinsamen Seite $ AD $ nach (K.6)
\begin{equation*}
\begin{aligned}
\triangle ABD \cong \triangle DCA.
\end{aligned}
\end{equation*}
Dann ist aber auch
\begin{equation*}
\begin{aligned}
AC \cong BD.
\end{aligned}
\end{equation*}
Sei nun weiter $ E \in \ray{BA} $ mit $ AB \cong AE $. Dann ist auch $ AE \parallel CD $.\\
Dann gilt wie oben
\begin{equation*}
\begin{aligned}
\triangle DBA \cong \triangle DCA \cong \triangle ACE
\end{aligned}
\end{equation*}
Insbesondere gilt damit
\begin{equation*}
\begin{aligned}
\angle DBA \simeq \angle CAE.
\end{aligned}
\end{equation*}
Nach (I.27) gilt dann
\begin{equation*}
\begin{aligned}
AC \parallel BD
\end{aligned}
\end{equation*}
\begin{flushright}
	$ \Box $
\end{flushright}
\section*{MC}
\begin{enumerate}
	\item falsch
	\item falsch
	\item falsch (?)
	\item richtig
	\item (?)
\end{enumerate}
\end{document}

