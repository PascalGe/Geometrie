\documentclass[11pt]{article}

\usepackage[utf8]{inputenc}
\usepackage[T1]{fontenc}
\usepackage[ngerman]{babel}
\usepackage{lmodern}
\usepackage[german=quotes]{csquotes}
\usepackage{amsmath}
\usepackage{amssymb}
\usepackage{amsthm}
\usepackage{graphicx}
\usepackage{scrpage2}
\usepackage{geometry}
\usepackage{tikz, tikz-3dplot}
\usepackage[bookmarks=true,
bookmarksopen=true,
bookmarksnumbered=false,
pdfstartpage=1,
baseurl=,
pdftitle={ },
pdfauthor={Pascal Gepperth},
pdfstartview={FitH},
pdfsubject={ },
pdfkeywords={ },
breaklinks=true,
colorlinks=true,
linkcolor=black,
anchorcolor=black,
citecolor=black,
filecolor=black,
menucolor=black,
pagecolor=black,
urlcolor=black ]{hyperref}
\newcommand{\de}{\mathrm{d}}
\newcommand{\dif}[1]{\frac{\mathrm{d}}{\mathrm{d}#1}}
\usepackage{enumerate}
\newcommand{\ray}[1]{\stackrel{\rightharpoonup}{#1}}
\usepackage{marginnote}
\let\marginpar\marginnote
\renewcommand*{\marginfont}{\footnotesize}
\renewcommand{\vec}{\textbf}

\geometry{left=3cm, right=3cm, top=3cm, bottom=3cm}

\pagestyle{scrheadings}
\ohead{Geometrie\\
	Blatt 11\\
	P.Gepperth, S.Jung\\
	Gruppe 4}

\begin{document}
\section*{Aufgabe 3}
Sei $ \triangle ABC \in \mathbb{C} $.
\subsection*{Behauptung}
\begin{enumerate}[i)]
	\item Der Schwerpunkt $ S $ von $ \triangle ABC $ ist
	\begin{equation*}
	\begin{aligned}
	S = \frac{1}{3} \left(A + B + C\right)
	\end{aligned}
	\end{equation*}
	\item Satz von Napoleon
\end{enumerate}
\subsection*{Beweis}
\begin{enumerate}[i)]
	\item Seien die Seitenmittelpunkte
	\begin{equation*}
	\begin{aligned}
	m_A &= \frac{1}{2} \left(B + C\right)\\
	m_B &= \frac{1}{2} \left(A + C\right)\\
	m_C &= \frac{1}{2} \left(A + B\right)
	\end{aligned}
	\end{equation*}
	Dann gilt
	\begin{equation*}
	\begin{aligned}
	\frac{\left|A - B\right|}{\left|m_C - B\right|} &= \frac{2}{1}\\
	&= \frac{\left|B - C\right|}{\left|m_A - B\right|}
	\end{aligned}
	\end{equation*}
	Mit dem ersten Strahlensatz (Zentrum in $ B $) folgt damit
	\begin{equation*}
	\begin{aligned}
	m_A - m_C &= \lambda\left(A - C\right)
	\end{aligned}
	\end{equation*}
	für ein $ \lambda \in \mathbb{R}^* $. Mit dem zweiten Strahlensatz folgt damit
	\begin{equation*}
	\begin{aligned}
	\frac{\left|A-C\right|}{\left|m_A - m_C\right|} &= \frac{2}{1}
	\end{aligned}
	\end{equation*}
	Man betrachte nun die Strahlensatzfigur mit Zentrum $ S $, so gilt
	\begin{equation*}
	\begin{aligned}
	\frac{2}{1} &= \frac{\left|A-C\right|}{\left|m_A - m_C\right|} = \frac{\left|A - S\right|}{\left|S - m_A\right|}
	\end{aligned}
	\end{equation*}
	Analog für die anderen Seiten. Damit folgt
	\begin{equation*}
	\begin{aligned}
	S &= m_A + \frac{1}{3} \left(C-m_C\right)\\
	&= \frac{1}{2} \left(A+B\right) + \frac{1}{3} \left(C - \frac{1}{2}\left(A + B\right)\right)\\
	&= \frac{1}{3} \left(C - \frac{1}{2}\left(A+B\right) + \frac{3}{2}\left(A+B\right)\right)\\
	&= \frac{1}{3} \left(A + B + C\right)
	\end{aligned}
	\end{equation*}
	\item 
\end{enumerate}

\end{document}

