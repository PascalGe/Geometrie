\documentclass[11pt]{article}

\usepackage[utf8]{inputenc}
\usepackage[T1]{fontenc}
\usepackage[ngerman]{babel}
\usepackage{lmodern}
\usepackage[german=quotes]{csquotes}
\usepackage{amsmath}
\usepackage{amssymb}
\usepackage{amsthm}
\usepackage{graphicx}
\usepackage{scrpage2}
\usepackage{geometry}
\usepackage{tikz, tikz-3dplot}
\usepackage[bookmarks=true,
bookmarksopen=true,
bookmarksnumbered=false,
pdfstartpage=1,
baseurl=,
pdftitle={ },
pdfauthor={Pascal Gepperth},
pdfstartview={FitH},
pdfsubject={ },
pdfkeywords={ },
breaklinks=true,
colorlinks=true,
linkcolor=black,
anchorcolor=black,
citecolor=black,
filecolor=black,
menucolor=black,
pagecolor=black,
urlcolor=black ]{hyperref}
\newcommand{\de}{\mathrm{d}}
\newcommand{\dif}[1]{\frac{\mathrm{d}}{\mathrm{d}#1}}
\usepackage{enumerate}
\newcommand{\ray}[1]{\stackrel{\rightharpoonup}{#1}}
\usepackage{marginnote}
\let\marginpar\marginnote
\renewcommand*{\marginfont}{\footnotesize}

\geometry{left=3cm, right=3cm, top=3cm, bottom=3cm}

\pagestyle{scrheadings}
\ohead{Geometrie\\
	Blatt 5\\
	P.Gepperth, S.Jung\\
	Gruppe 4}

\begin{document}
\section*{Aufgabe 1}
Es gilt
\begin{equation*}
\begin{aligned}
\emptyset \neq Y \text{ ist AUR} :\Leftrightarrow U_Y := \lbrace \overrightarrow{PQ} : Q \in Y \rbrace \text{ ist UVR für ein } P \in Y
\end{aligned}
\end{equation*}
\subsection*{Behauptung}
\begin{enumerate}[1)]
	\item
	\begin{enumerate}[i)]
		\item
		\begin{equation*}
		\begin{aligned}
		U_Y = \lbrace \overrightarrow{P'Q}: Q\in Y \rbrace \quad \forall P' \in Y
		\end{aligned}
		\end{equation*}
		\item
		\begin{equation*}
		\begin{aligned}
		Y = P' + U_Y \quad \forall P' \in Y
		\end{aligned}
		\end{equation*}
	\end{enumerate}
\item 
\begin{equation*}
\begin{aligned}
Y \in \mathbb{A}(V) \text{ ist AUR } \Rightarrow U_Y = \lbrace \overrightarrow{PQ}: P,Q \in Y \rbrace \subseteq V \text{ ist UVR}
\end{aligned}
\end{equation*}
Die Umkehrung gilt nicht.
\end{enumerate}
\subsection*{Beweis}
\begin{enumerate}[1)]
	\item
	\begin{enumerate}[i)]
		\item Sei $ P' \in Y $ beliebig. Dann gilt $ \overrightarrow{PP'} \in U_Y $ und es gilt
		\begin{equation*}
		\begin{aligned}
		&& \overrightarrow{PQ} &= \overrightarrow{PP'} + \overrightarrow{P'Q}\\
		\Leftrightarrow && \underbrace{\overrightarrow{PQ} - \overrightarrow{PP'}}_{\in U_Y} = \overrightarrow{P'Q} \in U_Y.
		\end{aligned}
		\end{equation*}
		Daraus folgt
		\begin{equation}
		\begin{aligned}
		\lbrace \overrightarrow{P'Q} : Q \in Y \rbrace \subseteq U_Y. \label{eq:Inklusion1}
		\end{aligned}
		\end{equation}
		Da $ P \in Y $ gilt $ \overrightarrow{P'P} \in \lbrace \overrightarrow{P'Q} : Q \in Y \rbrace $. Damit gilt nun
		\begin{equation*}
		\begin{aligned}
		&& \overrightarrow{P'Q} &= \overrightarrow{P'P} + \overrightarrow{PQ}\\
		\Leftrightarrow && \overrightarrow{P'Q} - \overrightarrow{P'P} &= \overrightarrow{PQ} \in 	\lbrace \overrightarrow{P'Q} : Q \in Y \rbrace.
		\end{aligned}
		\end{equation*}
		Damit gilt
		\begin{equation}
		\begin{aligned}
		\lbrace \overrightarrow{PQ} : Q \in Y \rbrace = U_Y \subseteq \lbrace \overrightarrow{P'Q} : Q \in Y \rbrace. \label{eq:Inklusion2}
		\end{aligned}
		\end{equation}
		Mit (\ref{eq:Inklusion1}) und (\ref{eq:Inklusion2}) gilt nun
		\begin{equation*}
		\begin{aligned}
		\lbrace \overrightarrow{P'Q} : Q \in Y \rbrace = U_Y
		\end{aligned}
		\end{equation*}
		\item 
		\begin{equation*}
		\begin{aligned}
		Y &= P + U_Y\\
		&= \lbrace \overrightarrow{OP} + u : u \in U_Y \rbrace\\
		&= \lbrace \overrightarrow{OP'} + \underbrace{\overrightarrow{P'P}}_{\in U_Y} + u : u \in U_Y \rbrace \\
		&= \lbrace \overrightarrow{OP'} + u : u \in U_Y \rbrace\\
		&= P' + U_Y
		\end{aligned}
		\end{equation*}
	\end{enumerate}
\item Es gilt
\begin{equation*}
\begin{aligned}
Y \in \mathbb{A}(V) \text{ ist AUR} &\stackrel{\mathrm{Def}}{\Leftrightarrow} U_Y = \lbrace \overrightarrow{PQ} : Q\in Y \rbrace \text{ für ein } P \in Y \text{ ist UVR}\\
&\stackrel{1.i)}{\Rightarrow} U_Y = \lbrace \overrightarrow{PQ} : Q\in Y \rbrace \quad \forall P \in Y \text{ ist UVR}\\
&\stackrel{}{\Leftrightarrow} U_Y = \lbrace \overrightarrow{PQ} : P,Q\in Y \rbrace \text{ ist UVR}.
\end{aligned}
\end{equation*}
Man betrachte nun einen Strahl $ Y = \ray{AB} \in \mathbb{A}(V) $, so ist
\begin{equation*}
\begin{aligned}
U_Y = \lbrace \overrightarrow{PQ} : P,Q\in Y \rbrace
\end{aligned}
\end{equation*}
ein UVR. Bekanntlich ist ein Strahl aber kein AUR.
\end{enumerate}
\begin{flushright}
	$ \Box $
\end{flushright}
\newpage
\section*{Aufgabe 3}
\begin{enumerate}[1)]
	\item Es sei $ L \subset \mathbb{R}^5 $ gegeben als Lösungsmenge von
	\begin{equation*}
	\begin{aligned}
	x_1 = 1, \quad x_2 = x_5, \quad x_3 = 1, \quad x_4 = 0, \quad x_1 + x_4 = 1
	\end{aligned}
	\end{equation*}
	und $ E $ als Bild der affinen Abbildung $ \mathbb{R}^3 \to \mathbb{R}^5 $
	\begin{equation*}
	\begin{aligned}
	(t_1,t_2,t_3) \mapsto (1,0,0,1,0) + t_1 (0,2,1,-1,2) + t_2(1,0,0,1,0) + t_3(1,2,1,0,2)
	\end{aligned}
	\end{equation*}.\\
	Dann ist
	\begin{equation*}
	\begin{aligned}
	L = (1,0,1,0,0) + t(0,1,0,0,1), \quad t \in \mathbb{R}
	\end{aligned}
	\end{equation*}
	die Parameterform von $ L $ mit $ \dim(L) = 1 $. Für die implizite Form sind $ \dim(\mathbb{R}^5) - \dim(L) = 4 $ Gleichungen nötig. Hierfür wähle man
	\begin{equation*}
	\begin{aligned}
	x_1 = 1, \quad x_2 = x_5, \quad x_3 = 1, \quad x_4 = 0,
	\end{aligned}
	\end{equation*}
	da $ x_1+x_4 = 1 $ hierdurch bereits erfüllt ist.\\
	Nun ist
	\begin{equation*}
	\begin{aligned}
	E = (1,0,0,1,0) + t_1 (0,2,1,-1,2) + t_2(1,0,0,1,0) + t_3(1,2,1,0,2), \quad t_1,t_2,t_3 \in \mathbb{R}
	\end{aligned}
	\end{equation*}
	die Parameterform von $ E $ mit $ \dim(E) = 3 $. Es gilt damit
	\begin{equation*}
	\begin{aligned}
	x_1 &= 1 + t_2 + t_3\\
	x_2 &= 2 (t_1 + t_3)\\
	x_3 &= t_1 + t_3\\
	x_4 &= 1 - t_1 + t_2\\
	x_5 &= 2(t_1 + x_3)
	\end{aligned}
	\end{equation*}
	Offenbar gilt
	\begin{equation}
	\begin{aligned}
	x_2 = x_5 \label{eq:impl_1}
	\end{aligned}
	\end{equation}
	und
	\begin{equation}
	\begin{aligned}
	x_1 - x_4 = x_3 \label{eq:impl_2}
	\end{aligned}
	\end{equation}
	Die Gleichungen (\ref{eq:impl_1}) und (\ref{eq:impl_2}) sind $ \dim(\mathbb{R}^5) - \dim(E) = 2 $ Gleichungen und beinhalten alle $ x_i $ für $ i=1,\dots,5 $ und sind damit die implizite Form.
	\item Man betrachte nun die impliziten Formen von $ L $
	\begin{equation*}
	\begin{aligned}
	x_1 = 1, \quad x_2 = x_5, \quad x_3 = 1, \quad x_4 = 0
	\end{aligned}
	\end{equation*}
	und $ E $
	\begin{equation*}
	\begin{aligned}
	x_2 = x_5, \quad x_1 - x_4 = x_3.
	\end{aligned}
	\end{equation*}
	Eingesetzt ergibt sich
	\begin{equation*}
	\begin{aligned}
	x_3 &= x_1 - x_4\\
	&= 1 - 0 = 1.
	\end{aligned}
	\end{equation*}
	$ E $ schränkt $ L $ folglich nicht weiter ein. Damit gilt
	\begin{equation*}
	\begin{aligned}
	E \cap L = L,
	\end{aligned}
	\end{equation*}
	$ L $ liegt also in $ E $. Ist dies der Fall, so gilt aber
	\begin{equation*}
	\begin{aligned}
	E \cup L = E
	\end{aligned}
	\end{equation*}
	mit Dimensionen wie oben.
\end{enumerate}


\section*{MC}
\begin{enumerate}
	\item falsch
	\item richtig
	\item falsch
	\item richtig
	\item falsch
\end{enumerate}
\end{document}

