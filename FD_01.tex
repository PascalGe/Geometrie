\documentclass[11pt]{article}

\usepackage[utf8]{inputenc}
\usepackage[T1]{fontenc}
\usepackage[ngerman]{babel}
\usepackage{lmodern}
\usepackage[german=quotes]{csquotes}
\usepackage{amsmath}
\usepackage{amssymb}
\usepackage{amsthm}
\usepackage{graphicx}
\usepackage{scrpage2}
\usepackage{geometry}
\usepackage{tikz, tikz-3dplot}
\usepackage[bookmarks=true,
bookmarksopen=true,
bookmarksnumbered=false,
pdfstartpage=1,
baseurl=,
pdftitle={ },
pdfauthor={Pascal Gepperth},
pdfstartview={FitH},
pdfsubject={ },
pdfkeywords={ },
breaklinks=true,
colorlinks=true,
linkcolor=black,
anchorcolor=black,
citecolor=black,
filecolor=black,
menucolor=black,
pagecolor=black,
urlcolor=black ]{hyperref}
\newcommand{\de}{\mathrm{d}}
\newcommand{\dif}[1]{\frac{\mathrm{d}}{\mathrm{d}#1}}
\usepackage{enumerate}
\newcommand{\ray}[1]{\stackrel{\rightharpoonup}{#1}}
\usepackage{marginnote}
\let\marginpar\marginnote
\renewcommand*{\marginfont}{\footnotesize}
\usepackage{siunitx}

\geometry{left=3cm, right=3cm, top=3cm, bottom=3cm}

\pagestyle{scrheadings}
\ohead{Pascal Gepperth}
\ifoot{\today}

\begin{document}

\section*{Sitzung 1}
\subsection*{Uni-Teil}
\subsubsection*{Erweiterung des ersten Strahlensatzes}
\begin{equation*}
\begin{aligned}
&& \frac{a}{c} &= \frac{a +b}{c + b}\\
\Leftrightarrow && \frac{c+d}{c} &= \frac{a+b}{a}\\
\Leftrightarrow && \frac{d}{c} &= \frac{b}{a}\\
\Leftrightarrow && \frac{a}{b} &= \frac{c}{d}
\end{aligned}
\end{equation*}
\subsubsection*{Zweiter Strahlensatz}
Es gilt der erste Strahlensatz.\\
\textbf{Skizze:}
\begin{center}
	\begin{tikzpicture}[scale = 2, thick]
		\coordinate (A) at (2,2);
		\coordinate (B) at (0,0);
		\coordinate (C) at (2,0);
		\coordinate (D) at (3,1);
		\coordinate (E) at (3,0);
		\coordinate (F) at (3,3);
		
		\draw (A)node[above]{$A$} -- (B)node[below]{$B$} -- (C)node[below]{$C$} -- (A) -- (F)node[above]{$F$} -- (E)node[below]{$E$} -- (C) -- (D)node[right]{$D$};
	\end{tikzpicture}
\end{center}
Es gilt mit dem ersten, erweiterten Strahlensatz und Zentrum $ B $
\begin{equation}
\begin{aligned}
\frac{BC}{CE} = \frac{BA}{AF} \label{eq:ZentrB}
\end{aligned}
\end{equation}
sowie für Zentrum $ E $
\begin{equation*}
\begin{aligned}
\frac{ED}{DF} = \frac{EC}{CB}
\end{aligned}
\end{equation*}
und somit natürlich auch
\begin{equation}
\begin{aligned}
\frac{DF}{ED} = \frac{BC}{CE} \label{eq:ZentrE}
\end{aligned}
\end{equation}
Aus (\ref{eq:ZentrB}) und (\ref{eq:ZentrE}) folgt dann
\begin{equation*}
\begin{aligned}
&&\frac{BA}{AF} &= \frac{DF}{ED}\\
\Leftrightarrow && \frac{BA}{DF} &= \frac{AF}{ED}
\end{aligned}
\end{equation*}
Da $ AF \parallel CD $ und $ AC \parallel DF $ ist $ ACDF $ ein Parallelogramm. Damit gilt $ AC = DF $ und folglich
\begin{equation*}
\begin{aligned}
&& \frac{BA}{DF} &= \frac{AF}{DE}\\
\Leftrightarrow && \frac{DE}{DF} &= \frac{AF}{BA}\\
\Leftrightarrow && \frac{DE}{DF} &= \frac{AF}{BA}\\
\Leftrightarrow && \frac{DE + DF}{DF} &= \frac{AF + BA}{BA}\\
\Leftrightarrow && \frac{BA}{DF} &= \frac{AF +BA}{DE + DF}\\
\Leftrightarrow && \frac{BA}{AC}&= \frac{BF}{EF}
\end{aligned}
\end{equation*}
\subsubsection*{Widerlegung der Umkehrung}
Betrachte die Punkte $ A(0,0), B(2,2), C(1,0), D(3,0) $.\\
\textbf{Skizze:}
\begin{center}
	\begin{tikzpicture}[scale = 2, thick]
		\coordinate (A) at (0,0);
		\coordinate (B) at (2,2);
		\coordinate (C) at (1,0);
		\coordinate (D) at (3,0);
		
		\draw (B)node[above]{$B$} -- (A)node[left]{$A$} -- (C)node[below]{$C$} -- (D)node[below]{$D$};
		\draw[dashed] (D) -- (B) -- (C);
	\end{tikzpicture}
\end{center}
Offenbar gilt
\begin{equation*}
\begin{aligned}
\frac{AB}{BC} &= \frac{AB}{BC},
\end{aligned}
\end{equation*}
doch die Geraden $ B\lor C $ und $ B \lor D $ sind verschieden, da $ C \neq D $, es gilt aber
\begin{equation*}
\begin{aligned}
B \in B\lor C \cap B\lor D \neq \emptyset.
\end{aligned}
\end{equation*}
Folglich sind die Geraden nicht parallel.
\newpage
\section*{Sitzung 2}
\subsection*{Schul-Teil}
\subsubsection*{Eigenschaften der Zentrischen Streckung (Elemente 5)}
Beispiele an Konstruktionen mit maßstäblicher Vergrößerung.
\paragraph{Definition}
Eine \textbf{zentrische Streckung} wird festgelegt durch das \textbf{Streckzentrum \textit{Z}} und den \textit{positiven} \textbf{Streckfaktor \textit{k}}.\\
Zu einem Punkt erhältst du den Bildpunkt wie folgt:
\begin{enumerate}[(1)]
	\item Wenn der Punkt $ P $ nicht mit dem Zentrum zusammenfällt, dann erhält man den Bildpunkt $ P' $ wie folgt:
	\begin{enumerate}[(a)]
		\item Zeichne die Halbgerade $ \ray{ZP} $.
		\item Zeichne den Punkt $ P' $ auf der Halbgeraden $ \ray{ZP} $ so, dass gilt
		\begin{align*}
		\left|ZP'\right| = k \cdot \left|ZP\right|
		\end{align*}
	\end{enumerate}
	\item Der Bildpunkt $ Z' $ von $ Z $ fällt mit $ Z $ zusammen: $ Z' = Z $.
\end{enumerate}
\paragraph{Zentrische Streckung mit negativem Streckfaktor}
Eingeführt als zentrische Streckung um $ \left|k\right| $ und anschließender Punktspiegelung in $ Z $.
\paragraph{Diskussion}
S.18, Afg 12
\paragraph{Eigenschaften}
Betrachtung verschiedener Abbildungen (vgl S.19) \marginpar{Abb. ungenau, nur intuitive Vermutungen}
\begin{enumerate}[(1)]
	\item Drehung + Verkleinerung
	\item Stauchung von Winkel
	\item Zentrische Streckung Dreieck
	\item Verschiedene Streckungsfaktoren
\end{enumerate}
mit Hinblick auf zentrische Streckungen. Beurteilung:
\begin{enumerate}[(1)]
 \item nicht, da $ AB $ nicht parallel zu $ A'B' $.
 \item nicht, da Winkel nicht erhalten.
 \item ja.
 \item nicht, da unterschiedliche Streckfaktoren.
 \end{enumerate}
\paragraph{Satz}
Für jede \textit{zentrische Streckung} mit einem positiven Streckfaktor $ k $ gilt:
\begin{enumerate}[(a)]
	\item Gerade und Bildgerade sind parallel.
	\item Bildstrecke ist k-mal so lang wie Originalstrecke.
	\item Winkel und Bildwinkel sind gleich groß.
\end{enumerate}
\paragraph{Beweis des Satzes}
\begin{enumerate}[(a)]
	\item Wird nicht bewiesen.
	\item Betrachtung zweier Fälle
	\begin{enumerate}
		\item Fall: Die Strecke $ AB $ liegt auf einer Geraden durch das Streckzentrum. \marginpar{$ Z \in A\lor B $}
		\item Fall:
	\end{enumerate}
	\item 
\end{enumerate}
\begin{flushright}
	$ \Box $
\end{flushright}
\subsubsection*{Vergleich der Schulbücher}
\end{document}

