\documentclass[11pt]{article}

\usepackage[utf8]{inputenc}
\usepackage[T1]{fontenc}
\usepackage[ngerman]{babel}
\usepackage{lmodern}
\usepackage[german=quotes]{csquotes}
\usepackage{amsmath}
\usepackage{amssymb}
\usepackage{amsthm}
\usepackage{graphicx}
\usepackage{scrpage2}
\usepackage{geometry}
\usepackage{tikz, tikz-3dplot}
\usepackage[bookmarks=true,
bookmarksopen=true,
bookmarksnumbered=false,
pdfstartpage=1,
baseurl=,
pdftitle={ },
pdfauthor={Pascal Gepperth},
pdfstartview={FitH},
pdfsubject={ },
pdfkeywords={ },
breaklinks=true,
colorlinks=true,
linkcolor=black,
anchorcolor=black,
citecolor=black,
filecolor=black,
menucolor=black,
pagecolor=black,
urlcolor=black ]{hyperref}
\newcommand{\de}{\mathrm{d}}
\newcommand{\dif}[1]{\frac{\mathrm{d}}{\mathrm{d}#1}}
\usepackage{enumerate}
\newcommand{\ray}[1]{\stackrel{\rightharpoonup}{#1}}
\usepackage{marginnote}
\let\marginpar\marginnote
\renewcommand*{\marginfont}{\footnotesize}
\renewcommand{\vec}{\textbf}

\geometry{left=3cm, right=3cm, top=3cm, bottom=3cm}

\pagestyle{scrheadings}
\ohead{Geometrie\\
	Blatt 10\\
	P.Gepperth, S.Jung\\
	Gruppe 4}

\begin{document}

\section*{Aufgabe 1}
Man betrachte $ \mathbb{P}_2(\mathbb{R}) = \mathbb{R}^2 \cup H_\infty $ und $ \mathbb{P}_1(\mathbb{C}) = \mathbb{C} \cup \lbrace \infty \rbrace $ mit den Standardkarten, also $ H_\infty : X_3 = 0, \infty = \left[1:0\right] $.
\subsection*{Behauptung}
\begin{enumerate}
	\item $ \pi: \mathbb{P}_2(\mathbb{R}) \to \mathbb{P}_1(\mathbb{C}), \left[X_1:X_2:X_3\right] \mapsto \left[X_1 + iX_2 : X_3\right] $ ist wohldefiniert und surjektiv.
	\item $ \pi $ induziert Bijektion der Standardkarte $ \mathbb{R}^2 \equiv \lbrace X_3 \neq 0 \rbrace $ auf $ \mathbb{C} \equiv \lbrace Z_2 \neq 0 \rbrace $.
	\item $ \pi^{-1}(\infty) = H_\infty $.
\end{enumerate}
\subsection*{Beweis}
\begin{enumerate}
	\item  Es seien
	\begin{equation*}
	\begin{aligned}
	\left[X_1:X_2:X_3\right] = \left[X_1':X_2':X_3'\right],
	\end{aligned}
	\end{equation*}
	dann gilt
	\begin{equation*}
	\begin{aligned}
	X_1' &= \lambda X_1\\
	X_2' &= \lambda X_2\\
	X_3' &= \lambda X_3
	\end{aligned}
	\end{equation*}
	für ein $ \lambda \in \mathbb{R}\backslash\lbrace0\rbrace $. Dann ist
	\begin{equation*}
	\begin{aligned}
	\pi \left(\left[X_1:X_2:X_3\right]\right) = \left[X_1+iX_2 : X_3\right]
	\end{aligned}
	\end{equation*}
	und
	\begin{equation*}
	\begin{aligned}
	\pi\left(\left[X_1':X_2':X_3'\right]\right) &= \left[X_1'+iX_2':X_3'\right]\\
	&= \left[\lambda X_1+i\lambda X_2:\lambda X_3'\right]\\
	&=\left[X_1+iX_2 : X_3\right]\\
	&= \pi \left(\left[X_1:X_2:X_3\right]\right).
	\end{aligned}
	\end{equation*}
	Folglich ist $ \pi $ wohldefiniert.\\
	Sei nun weiter $ \left[Z : X_3\right] \in \mathbb{P}_1(\mathbb{C}) $ beliebig, mit $ Z \in \mathbb{C} $. Wähle
	\begin{equation*}
	\begin{aligned}
	X_1 &= \mathrm{Re}(Z) \in \mathbb{R}\\
	X_2 &= \mathrm{Im}(Z) \in \mathbb{R},
	\end{aligned}
	\end{equation*}
	so gilt
	\begin{equation*}
	\begin{aligned}
	\pi\left(\left[X_1:X_2:X_3\right]\right) &= \left[X_1+iX_2:X_3\right]\\
	&= \left[Z: X_3\right].
	\end{aligned}
	\end{equation*}
	Somit ist $ \pi $ surjektiv.
	\paragraph{Anmerkung} o.B.d.A. ist $ X_3 \in \mathbb{R} $, da sonst
	\begin{equation*}
	\begin{aligned}
	\left[Z :X_3\right] = \left[Z/X_3 : 1\right]
	\end{aligned}
	\end{equation*}
	\item Sei $ X_3 \neq 0 $, so ist o.B.d.A $ X_3 = 1 $ (siehe Anmerkung). Sei nun
	\begin{equation*}
	\begin{aligned}
	\pi\left(\left[X_1:X_2:1\right]\right) = \pi\left(\left[X_1':X_2':1\right]\right),
	\end{aligned}
	\end{equation*}
	dann gilt
	\begin{equation*}
	\begin{aligned}
	\left[X_1+iX_2 :1 \right] &= \pi\left(\left[X_1:X_2:1\right]\right)\\
	&= \pi\left(\left[X_1':X_2':1\right]\right)\\
	&= \left[X_1'+iX_2' :1 \right].
	\end{aligned}
	\end{equation*}
	Dann gilt aber
	\begin{equation*}
	\begin{aligned}
	\left[X_1'+iX_2' :1 \right] &= \lambda \left[X_1+iX_2 :1 \right]\\
	&= \left[\lambda X_1+i \lambda X_2 :\lambda \right]
	\end{aligned}
	\end{equation*}
	für ein $ \lambda \in \mathbb{R}\backslash\lbrace0\rbrace $. O.B.d.A. sei $ \lambda = 1 $ (siehe Anmerkung), dann gilt
	\begin{equation*}
	\begin{aligned}
		\left[X_1'+iX_2' :1 \right] &= \left[X_1+iX_2 :1 \right]
	\end{aligned}
	\end{equation*}
	und damit (Vergleich letzter Komponente)
	\begin{equation*}
	\begin{aligned}
	X_1'+iX_2' = X_1+iX_2.
	\end{aligned}
	\end{equation*}
	Dies ist nur erfüllt, wenn
	\begin{equation*}
	\begin{aligned}
	X_1' &= X_1\\
	X_2' &= X_2.
	\end{aligned}
	\end{equation*}
	Damit folgt aber direkt
	\begin{equation*}
	\begin{aligned}
	\left[X_1:X_2:1\right] = \left[X_1':X_2':1\right],
	\end{aligned}
	\end{equation*}
	also ist $ \pi $ auf den Standardkarten injektiv. Mit oben gezeigter Surjektivität besteht hier also eine Bijektion.
	\item Es gilt
	\begin{equation*}
	\begin{aligned}
	\pi^{-1}\left(\infty\right) &= \pi^{-1}\left(\left[1:0\right]\right)\\
	&=\pi^{-1}\left(\left[z:0\right]\right)\ \forall z \in \mathbb{C}\\
	&=\left[\mathrm{Re}(z):\mathrm{Im}(z):0\right]\ \forall z \in \mathbb{C}\\
	&=\left[x:y:0\right]\ \forall x,y \in \mathbb{R}\\
	&= H_\infty
	\end{aligned}
	\end{equation*}
\end{enumerate}

\end{document}

